% arara: lualatex: { shell: yes }
% arara: biber
% arara: lualatex: { shell: yes }
% arara: lualatex: { shell: yes }
% arara: lualatex: { shell: yes, synctex: yes }
\documentclass[aspectratio=169]{beamer}

% LuaLaTeX fixes
\usepackage{luatex85}

% Math support
\usepackage{mathtools}
\usepackage{empheq}

% Beamer settings
\PassOptionsToPackage{no-math}{fontspec}
\setbeamertemplate{navigation symbols}{}

% Fonts
\usepackage{microtype}
\usepackage{arevmath}
\usepackage{fontspec}
\setsansfont[
BoldFont={Fira Sans SemiBold},
ItalicFont={Fira Sans BookItalic},
BoldItalicFont={Fira Sans SemiBold Italic}
]{Fira Sans Book}
\setmonofont{DejaVu Sans Mono}[Scale=MatchLowercase]

% Language support
\usepackage{polyglossia}
\setmainlanguage[variant=usmax]{english}
\usepackage{csquotes}

% Links
\usepackage{hyperref}

% Tables
\usepackage{booktabs}

% Captions
\usepackage{caption}

% PGFPlots support
\usepackage{pgfplots}
\pgfplotsset{compat=1.16,grid style=dashed,enlargelimits=auto}

% Cache figures
\usepgfplotslibrary{external}
\tikzexternalize[prefix=cache/]

% Use colorbrewer
\usepgfplotslibrary{colorbrewer}
\pgfplotsset{
  % initialize Dark2-8:
  cycle list/Dark2-8,
  % combine it with ’mark list*’:
  cycle multiindex* list={
    mark list*\nextlist
    Dark2-8\nextlist
  },
}

% Support units
\usepackage{siunitx}
\usepgfplotslibrary{units}
\makeatletter
\pgfplotsset{
  unit code/.code 2 args =
  \begingroup
  \protected@edef\x{\endgroup\si{#2}}\x
}
\makeatother

% Support group plots
\usepgfplotslibrary{groupplots}

% Define data folder
\pgfplotsset{table/search path={data/}}

% Source code highlighting
\usepackage[newfloat]{minted}
\usepackage[many,minted]{tcolorbox}
\tcbset{shield externalize}

% Colors
\usepackage{xcolor}

\definecolor{monokai-background}{HTML}{282828} % from https://github.com/kevinsawicki/monokai
\tcbset{myminted/juliacode/.style = {%
    enhanced jigsaw, breakable, colback = monokai-background, boxrule = 0pt, frame hidden,
    listing only, listing engine=minted, %
    minted language = {lexer.py:Julia1Lexer -x},%
    minted options = {breaklines,escapeinside=||,mathescape,autogobble,fontsize=\footnotesize},%
    minted style = monokai, %
  }}

% Highlighting of math terms
\makeatletter
\def\mathcolor#1#{\@mathcolor{#1}}
\def\@mathcolor#1#2#3{%
  \protect\leavevmode
  \begingroup
    \color#1{#2}#3%
  \endgroup
}
\makeatother

% Change graphics path (for SVG image)
\graphicspath{{figures/}}

% \usepackage{tikzscale}

% Bibliography
\usepackage[sortcites,doi=false,isbn=false,url=false,style=authoryear,citestyle=authoryear-comp]{biblatex}
\bibliography{references.bib}

% Metadata
\title{Solving Delay Differential Equations with Julia}
\subtitle{DelayDiffEq}
\author{David Widmann}
\date{23 July 2019}
\institute{JuliaCon 2019}
\titlegraphic{%
  \includegraphics[height=1.5cm]{figures/TUM_logo.pdf}%
  \hspace{1.5cm}~%%
  \includegraphics[height=1.5cm]{figures/UU_logo.pdf}%
}

\begin{document}

\maketitle

\section{Introduction}

\begin{frame}{How it started}
  \begin{center}
    \includegraphics[height=0.8\textheight]{figures/github_issue.png}
  \end{center}
\end{frame}

\begin{frame}{Ordinary differential equations (ODEs)}
  \begin{equation*}
    x'(t) = f(t, x(t))
  \end{equation*}

  \begin{block}<2->{Implicit assumption}
    Derivative $x'(t)$ depends only on $t$ and the current state $x(t)$
  \end{block}

  \begin{alertblock}<3->{Problem}
    In nature many changes do not occur instantaneously
  \end{alertblock}
\end{frame}

\begin{frame}{Delay differential equations (DDEs)}
  A DDE is an ODE in which the derivative depends on past values of the state.

  \only<1>{
    \begin{equation*}
      \phantom{x'(t) = f(t, x(t), x(t - \tau_1), \ldots, x(t - \tau_n), x'(t - \tau_{n+1}), \ldots, x'(t - \tau_{n+m}))}
    \end{equation*}%
  }

  \only<2>{
    \begin{equation*}
      x'(t) = f(t, x(t), x(t - \tau))
    \end{equation*}
  }

  \only<3>{
    \begin{equation*}
      x'(t) = f(t, x(t), x(t - \tau(t)))
    \end{equation*}
  }

  \only<4>{
    \begin{equation*}
      x'(t) = f(t, x(t), x(t - \tau(t, x(t)))
    \end{equation*}
  }

  \only<5>{
    \begin{equation*}
      x'(t) = f(t, x(t), x(t - \tau_1), \ldots, x(t - \tau_n))
    \end{equation*}
  }

  \only<6>{
    \begin{equation*}
      x'(t) = f(t, x(t), x(t - \tau_1), \ldots, x(t - \tau_n), x'(t - \tau_{n+1}), \ldots, x'(t - \tau_{n+m}))
    \end{equation*}
  }
\end{frame}

\begin{frame}{Example: Population growth models}
  Let $N(t)$ be the population size of a single species at time $t$.

  \pause

  \begin{block}{Verhulst model (Logistic equation)}
    \begin{equation*}
      N'(t) = r N(t) (1 - N(t) / K),
    \end{equation*}
    where $r$ is the growth rate and $K$ is the carrying capacity.
  \end{block}

  \pause

  \begin{block}{Hutchinson's equation \parencite{hutchinson48_circul_causal_system_in_ecolog} (Delay logistic equation)}
    \begin{equation*}
      N'(t) = r N(t) (1 - N(t-\tau) / K)
    \end{equation*}
  \end{block}
\end{frame}

\begin{frame}{History function}
  \begin{block}{ODEs}
    Initial value problem (IVP) is of the form
    \begin{align*}
      x'(t) &= f(t, x(t)), && t \geq t_0,\\
      x(t_0) &= x_0.
    \end{align*}
  \end{block}

  \pause

  \begin{block}{DDEs}
    Initial value problem is generally of the form
    \begin{align*}
      x'(t) &= f(t, x(t), x(t - \tau_1), \ldots, x(t - \tau_n)), && t \geq t_0,\\
      x(t) &= x_0(t), && t \leq t_0.
    \end{align*}

    The function $x_0$ is called the \alert{history function} (or initial function).
  \end{block}
\end{frame}

\begin{frame}{Propagated discontinuities}
  \begin{itemize}[<+->]
  \item Even if $x_0$ is smooth, we usually have
    \begin{equation*}
      \lim_{t \uparrow t_0} x'_0(t) \neq \lim_{t \downarrow t_0} x'(t)
    \end{equation*}
    i.e., there exists a jump derivative discontinuity at $t_0$.

  \item Discontinuities propagate in time \parencite[see, e.g.,][]{Bellen_2003}
  \end{itemize}
\end{frame}

\begin{frame}{Example: Propagated discontinuities}
  \begin{align*}
    x'(t) &= -x(t - 1/3) - x(t - 1/5), && t \geq 0,\\
    x(t) &= \delta(t), && t \leq 0.
  \end{align*}

  \begin{figure}[htbp]
    \begin{center}
      \tikzsetnextfilename{discontinuities}
      \pgfplotsset{width=0.9\textwidth,height=0.7\textheight}
      \input{figures/generated/discontinuities_2delays.tex}
    \end{center}
  \end{figure}
\end{frame}

\begin{frame}{Dynamical structure}
  \begin{block}{ODEs}
    \begin{itemize}
    \item Bounded solutions of autonomous ODEs can only oscillate if there are at least two components
    \item Chaotic solutions only if there are at least three components
    \end{itemize}
  \end{block}

  \pause

  \begin{block}{DDEs}
    Oscillatory and even chaotic behaviour can occur in the scalar case
  \end{block}
\end{frame}

\begin{frame}{Example: Mackey-Glass equation}
  Model of circulating blood cells $x(t)$ at time $t$ by \textcite{mackey77_oscil_chaos_physiol_contr_system}, given by
  \begin{align*}
    x'(t) &= \frac{\beta x(t-\tau)}{1 + {x(t-\tau)}^n} - \gamma x(t), \quad &&t \geq 0,\\
    x(t) &= x_0(t), \quad &&t \in [-\tau, 0].
  \end{align*}
\end{frame}

\begin{frame}{Example: Solution of Mackey-Glass equation}
  \begin{figure}[hbt]
    \begin{center}
      \pgfplotsset{width=0.45\textwidth,height=0.7\textheight}
      \tikzsetnextfilename{mackey_glass}%
      \input{figures/generated/mackey_glass.tex}%
      \tikzsetnextfilename{mackey_glass_emb}%
      \input{figures/generated/mackey_glass_embedding.tex}
      \caption{\label{fig:mackey_glass}Solution and time delay embedding of Mackey-Glass equation for $0 \leq t \leq 600$ with $t_0 = 0$, $x_0 \equiv 0.5$, $\beta = 2$, $n = 9.65$, and $\tau = 2$ (parameter values taken from~\textcite{glass10_mackey_glass_equat}).}
    \end{center}
  \end{figure}
\end{frame}

\begin{frame}[fragile]{Formulating a DDE in DelayDiffEq}
  \begin{align*}
    x'(t) &= - x(t - 1), && t \in [0, 10], \\
    x(t) &= 1, && t \leq 0.
  \end{align*}

  \pause

  \begin{tcblisting}{myminted/juliacode}
    # DDE model
    f(u, h, p, t) = - h(p, t - 1)
    # history function
    h(p, t) = 1.0
    # initial value
    u₀ = 1.0
    # integration interval
    tspan = (0.0, 10.0)

    prob = DDEProblem(f, u₀, h, tspan)
  \end{tcblisting}
\end{frame}

\section{Solving simple DDEs analytically}

\begin{frame}{Method of Steps}
  Consider a DDE with one constant delay:
  \begin{align*}
    x'(t) &= f(t, x(t), x(t-\tau)), && t \geq t_0,\\
    x(t) &= x_0(t), && t \in [t_0 - \tau, t_0],
  \end{align*}

  \pause

  \begin{block}{Solving the DDE by a sequence of ODEs}
    \begin{itemize}[<+->]
    \item For $n = 1,2,\ldots$, compute a solution $x_n$ of the ODE
      \begin{align*}
        x'(t) &= f(t, x(t), x_{n-1}(t-\tau)), && t \in [t_{n-1}, t_{n-1} + \tau],\\
        x(t_{n-1}) &= x_{n-1}(t_{n-1}),
      \end{align*}
      where $t_n \coloneqq t_0 + n \tau$.
    \item Then $x$ defined by
      \begin{equation*}
        x(t) \coloneqq x_n(t), \qquad t \in [t_{n-1}, t_n],
      \end{equation*}
      is a solution to the DDE.
    \end{itemize}
    \end{block}
\end{frame}

\begin{frame}{Implications}
  \begin{block}{Consequence}
    Results from ODE theory such as existence, uniqueness, and boundedness of solutions can be applied
  \end{block}

  \pause

  \begin{block}{Note}
    Holds even for multiple state-dependent delays as long as they are uniformly strictly greater than $0$!
  \end{block}
\end{frame}

\begin{frame}{OrdinaryDiffEq}
  \begin{itemize}[<+->]
  \item Large number of highly performant native Julia solvers
    \begin{itemize}[<1->]
    \item Classic algorithms and ones from recent research
    \item For stiff and non-stiff problems
    \item \alert{Dense interpolants}
    \end{itemize}
  \item Support of arbitrary number types such as dual numbers, arbitrary precision numbers, and numbers with units
  \item Additional functionality such as parameter estimation, sensitivity analysis, bifurcation analysis, and integration with Flux (neural ODEs) provided by other Julia packages
  \end{itemize}
\end{frame}

\begin{frame}[fragile]{General setup in DelayDiffEq}
  \begin{itemize}[<+->]
  \item Setup a dummy ODE solver for solving $f_0(u, p, t) \coloneqq f(u, h, p, t)$
  \item Wrap the history function $h$, the dense solution of the dummy solver, and its current integrator in an instance $\tilde{h}$ of
    \begin{tcblisting}{myminted/juliacode}
      struct HistoryFunction{F1,F2,F3<:ODEIntegrator} <: Function
        h::F1
        sol::F2
        integrator::F3
      end
    \end{tcblisting}
  \item Create the ``true'' ODE solver for solving $f_c(u, p, t) \coloneqq f(u, \tilde{h}, p, t)$
  \item Solve the ODE with the true solver and ensure that the dummy ODE solver's value are updated after each step to match the true ODE solver
  \end{itemize}
\end{frame}

\begin{frame}{Difficulties}

  \begin{block}{Discontinuities}
    Solution should be sufficiently smooth in each step \parencite[see][]{Bellen_2003}
  \end{block}

  \pause

  \begin{block}{Unconstrained time stepping}
    Inefficient if the step is constrained to time steps that are small relative to the integration interval
  \end{block}

  \pause

  \begin{block}{Dependent delays}
    Does not work for arbitrary time- and state-dependent delays
  \end{block}
\end{frame}

\begin{frame}{What we do in DelayDiffEq}
  \begin{block}{Discontinuities}
    \begin{itemize}
    \item Track discontinuities up to the order of the solver
    \item Hit these discontinuities by using existing functionality in OrdinaryDiffEq
    \end{itemize}
  \end{block}

  \pause

  \begin{block}{Unconstrained time stepping}
    If the delay is smaller than the time step, we perform fixed-point iteration
  \end{block}

  \pause

  \begin{block}{Dependent delays}
    If a step is rejected, we try to find dependent delays in the current time step
  \end{block}
\end{frame}

\begin{frame}{Example: Quorum Sensing (QS) model}
  \vspace*{-1.5ex}
  \begin{figure}[hbp]
    \begin{center}
      \tikzsetnextfilename{qs_comparison}
      \input{figures/generated/qs_comparison.tex}
      \vspace*{-1.5ex}
      \caption{\label{fig:qs_comparison}Numerical simulations of the QS model with parameter estimates of experiment CC2~\parencite{buddrus-schiemann14_analy_n_acylh_lacton_dynam}, using different method of steps algorithms.}
    \end{center}
  \end{figure}
\end{frame}

\begin{frame}{Example: Waltman's model}

  Model of antibody production by \textcite{waltman1978}, given by
  \begin{align*}
    x_1'(t) &= - r x_1(t) x_2(t) - s x_1(t) x_4(t), \\
    x_2'(t) &= - r x_1(t) x_2(t) + \alpha r x_1(x_5(t)) x_2(x_5(t)) H(t - t_0), \\
    x_3'(t) &= r x_1(t) x_2(t), \\
    x_4'(t) &= - s x_1(t) x_4(t) - \gamma x_4(t) + \beta r x_1(x_6(t)) x_2(x_6(t)) H(t - t_1), \\
    x_5'(t) &= H(t - t_0) \frac{x_1(t) x_2(t) + x_3(t)}{x_1(x_5(t)) x_2(x_5(t)) + x_3(x_5(t))}, \\
    x_6'(t) &= H(t - t_1) \frac{10^{-12} + x_2(t) + x_3(t)}{10^{-12} + x_2(x_6(t)) + x_3(x_6(t))},
  \end{align*}
  where $H$ is the Heavyside step function with $H(t) = 0$ if $t < 0$ and $H(t) = 1$ if $t \geq 0$.
\end{frame}

\begin{frame}{Example: Simulation of Waltman's model}
  \begin{figure}[htbp]
    \begin{center}
      \tikzsetnextfilename{waltman}
      \input{figures/generated/waltman.tex}
      \caption{\label{fig:waltman}Numerical simulations of Waltman's model using \texttt{Rosenbrock23} (dashed lines) and RADAR5 (solid lines) with parameters $\alpha = 1.8$, $\beta = 20$, $\gamma = 0.002$, $r = 5 \times 10^4$, $s = 10^5$, $t_0 = 35$, $t_1 = 197$, and $x(t) = [0.5 \times 10^{-5}, 10^{-15}, 0, 0, 0, 0]$ for $t \leq 0$.}
    \end{center}
  \end{figure}
\end{frame}

\begin{frame}{Summary}
  \begin{block}{Current features}
    \begin{itemize}[<+->]
    \item Solvers for both non-stiff and stiff DDEs
    \item Discontinuity tracking for dependent delays
    \item Support for neutral DDEs
    \item Event handling
    \item Additional functionality such as parameter estimation and integration with Flux provided by external packages
    \end{itemize}
  \end{block}

  \begin{block}<+->{Future directions}
    \begin{itemize}
    \item Use of Anderson Acceleration for fixed-point iterations
    \item<+-> More extensive benchmarking, also against existing DDE solvers
    \end{itemize}
  \end{block}
\end{frame}

\begin{frame}{Acknowledgements}
  \begin{columns}[t]
    \column{0.5\textwidth}
    All contributors of DelayDiffEq, namely
    \begin{itemize}
    \item Chris Rackauckas
    \item Yingbo Ma
    \item Kanav Gupta
    \item Takafumi Arakaki
    \item Elliot Saba
    \item Anshul Singhvi
    \item Alfonso Landeros
    \item Vaibhav Kumar Dixit
    \item Scott P.\ Jones
    \item femtocleaner
    \end{itemize}

    \pause

    \column{0.5\textwidth}
    All contributors of JuliaDiffEq

    \begin{figure}
      \begin{center}
        \includegraphics[width=0.5\linewidth]{figures/juliadiffeq.png}
      \end{center}
    \end{figure}
  \end{columns}
\end{frame}

\appendix

\begin{frame}[fragile,allowframebreaks]{References}
  \printbibliography[heading=none]
\end{frame}

\end{document}
